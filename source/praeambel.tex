\documentclass[11pt,a4paper]{scrreprt}

\usepackage[utf8]{inputenc}
\usepackage[T1]{fontenc}
\usepackage{lmodern}
\usepackage{ngerman}

\usepackage{amssymb} % für Mengensymbole \mathbb{<symbol>}; \checkmark
\usepackage{amsmath} % \align
\usepackage[usenames,dvipsnames,svgnames,table]{xcolor}
\usepackage{caption} % \caption*{}
\usepackage{tabu} % requires \usepackage{longtable} \longtabu
\usepackage{longtable}
\usepackage{wasysym} % smileys \smiley \frownie

\usepackage[pdftex]{graphicx}
\graphicspath{{../Abbildungen/}}

\usepackage{tabularx}

\definecolor{myLinkColor}{rgb}{0.25882353,0.36470588,0.60392157}
\usepackage[
    pdftex,
    a4paper,
    bookmarks,
    bookmarksopen=true,
    bookmarksnumbered=true,
    pdfauthor={Kajetan Weiß},
    pdftitle={Digitaltechnik Aufarbeitung},
    colorlinks,
    linkcolor=myLinkColor,
    urlcolor=myLinkColor
]{hyperref}

\setlength{\parindent}{0em}
\setlength{\parskip}{1ex}

% em Breite des Geviertstrichs im aktuellen Zeichensatz (entspricht nicht der Breite des Buchstabens M siehe Geviert)
% ex Höhe des buchstabens "x" im aktuellen Zeichensatz 